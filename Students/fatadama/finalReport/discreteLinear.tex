\section{The Linear Discrete Kalman Filter}

The original Kalman filter considers a linear discrete dynamic model and a linear discrete measurement model.

\subsection{System Models}
The discrete dynamic model relates the state at some time $ k $ to the state at the current time $ k+1 $.
This is also known as the process model.

\begin{equation}
\xkone=\Phik\xk+\Gammak\uk + \Upsilonk\wk
\end{equation}
$\xk$ is some state of interest at time k.
$\uk$ is a deterministic input vector.
$\wk $ is zero-mean Guassian white-noise process with 
\begin{equation}
E[\wk\wk]=\Qk 
\end{equation}
\begin{equation}
E[\wk\w(j)]=\mathbf{0}  \quad j \neq k
\end{equation}

It is useful to find the first and second moments of these models.
\begin{equation}
E[\xkone]=\xprior\kone  =\Phik\xkhat+\Gammak\uk
\end{equation}
\begin{equation}
Cov[\xkone,\xkone]=\Pxx\prior\kone =\Phik\Pxx\k\Phik\trans +\Upsilonk\Qk\Upsilonk\trans 
\end{equation}

$ \xprior\kone $ and $ \Pxx\prior\kone $ can be used to predict the state at time $ k+1 $ given the state at $ k$.

A measurement vector $ \yktilde $ is available.
$ \yktilde $ relates to the state according to 

\begin{equation}
\yktilde=\Hk\xk+\vk
\end{equation}
\begin{equation}
E[\vk\vk]=\Rk 
\end{equation}

A measurement estimate $ \ykhat $ is found by taking the expectation of this model

\begin{equation}
E[\yktilde]=\hat\y\k=\Hk\xprior\k
\end{equation}

It is also useful to define and determine the second moments of this model
\begin{equation}
Cov[\yktilde,\yktilde]=\Pyy\k=\Hk\Pxx\prior\k\Hk\trans+\Rk
\end{equation}
\begin{equation}
Cov[\xk,\yktilde]=\Pxy\k=\Pxx\prior\k\Hk
\end{equation}

\subsection{Minimum Mean Square}
When a measurement $ \yktilde $ becomes available, the process model can be used to predict, or propagate, the state estimate from the previous time to the current time.
This is known as the prior estimate $ \xprior\k $.
It is desirable to update the state estimate $ \xprior\k $ using the information about the state provided by the measurement. 
This updated measurement, $ \xpost\k $, is known as the posterior estimate.

The update equations will first be developed by assuming an estimator in the form of a sequential linear estimator with a gain $ \Kk $ and then finding $ \Kk $ to minimize the mean square error (MSE).
It will then be shown that if the initial state is Guassian then this estimator is also the minimum mean square error (MMSE) estimator.

The updated state $ \xpost\k $ estimator is assumed to be in the form of a linear sequential estimator. 
\begin{equation}
\xpost\k = \xprior\k + \Kk\big(\yktilde-\ykhat\big)
\end{equation}

It is necessary to find $ \Kk $ so that $ \xhat\k\post $ minimizes some performance criteria. 
The mean square error is a natural choice. 
\begin{equation}
 MSE=\Pxx\post\k=E\big[\big(\xk-\xpost\k\big)\big(\xk-\xpost\k\big)\trans\big]
\end{equation}

By using the previously defined value of  $ \xprior\k $, the error $ \big(\xk-\xpost\k\big) $ can be written as
\begin{equation}
\xk-\xpost\k = \terma \big(\xk-\xkhat\big) + \Kk\vk
\end{equation}

After expanding out the terms of the square error and finding its expected value
\begin{equation}
\Pxx\post\k=\terma \Pxx\prior\k \terma \trans+\Kk\Rk\Kk
\end{equation}

To minimize MSE, it is equivalent to minimize the trace of $\Pxx\post$.
\begin{equation}
J=trace(\Pxx\post)
\end{equation}

Finding the partial derivative of $ J $ with respect to $ \Kk $
\begin{equation}
\frac{\partial J}{\partial \Kk} = 0 = -2\terma \Pxx\prior\k\Hk\trans+2\Kk\Rk
\end{equation}

Solving for $\Kk$
\begin{equation}
\Kk =\Pxx\prior\k\Hk\trans\big(\Hk\Pxx\prior\k\Hk\trans+\Rk\big)^{-1} 
\end{equation}

This choice of $\Kk $ minimizes the MSE for a linear estimator.
That is, the Kalman filter provides the optimal \textit{linear} MMSE estimate.

If the initial state is sampled from a normal distribution, the state remains linear because a Gaussian random variable that undergoes linear transformations remains linear. 
In this case, $\xk$ and $\yktilde$ are jointly Guassian

The MMSE estimator is known to be 
\begin{equation}
\xhat_{MMSE}\k=E[\xk|\yktilde]
\end{equation}

It is well known that for jointly Gaussian $ \xk $ and $ \yktilde $ 
\begin{equation}
E[\xk|\yktilde]=E[\xk]+\Pxy\k\Pyy^{-1}\k\big(\yktilde-E[\yktilde]\big)
\end{equation}

Having previously defined each term of this equation in terms of the process and measurement model, it can be re-written as 
\begin{equation}
E[\xk|\yktilde]=\xprior\k+\bigg(\Pxx\prior\k\Hk\trans\big(\Hk\Pxx\prior\k\Hk\trans+\Rk\big)^{-1}\bigg) \big(\yktilde-\Hk\xprior\k\big)
\end{equation}
\begin{equation}
E[\xk|\yktilde]=\xprior\k+\Kk \big(\yktilde-\Hk\xprior\k\big)
\end{equation}

Thus, for a Gaussian state prior, the Kalman filter is the MMSE estimator. 

\subsection{Algorithm Summary}

The linear discrete Kalman filter can be described and implemented in three distinct stages.

Initialize:
\begin{equation}
\xhat(t_0)=\xhat_0 
\end{equation}
\begin{equation}
\Pxx(t_0)=Cov[\xhat_0,\xhat_0]
\end{equation}

Predict:
\begin{equation}
\xprior\kone  =\Phik\xkhat+\Gammak\uk
\end{equation}
\begin{equation}
\Pxx\prior\kone =\Phik\Pxx\k\Phik\trans +\Upsilonk\Qk\Upsilonk\trans
\end{equation}

Update:
\begin{equation}
\Kk =\Pxx\prior\k\Hk\trans\big(\Hk\Pxx\prior\k\Hk\trans+\Rk\big)^{-1}
\end{equation}
\begin{equation}
\xpost\k = \xprior\k + \Kk\big(\yktilde-\ykhat\big) 
\end{equation}
\begin{equation}
\Pxx\post\k=\big(\mathrm{I}-\Kk\Hk\big)\Pxx\prior\k
\end{equation}

Using the models:
\begin{equation}
\xkone=\Phik\xk+\Gammak\uk + \Upsilonk\wk \qquad \wk\sim N\big(\mathbf{0},\Qk\big)
\end{equation}
\begin{equation}
\yktilde=\Hk\xk+\vk \qquad \vk\sim N\big(\mathbf{0},\Rk\big)
\end{equation}

This section has introduced the linear Kalman Filter and justified its optimality in a MMSE sense.
The next section presents a numerical implementation of the Kalman Filter.